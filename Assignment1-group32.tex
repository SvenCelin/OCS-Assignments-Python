\documentclass[12pt]{article}         %% What type of document you're writing.


%%%%% Preamble

%% Packages to use
\usepackage[margin=1.1in]{geometry}
\usepackage{amsmath,amsfonts,amssymb}   %% AMS mathematics macros
\usepackage[utf8]{inputenc}
\usepackage{chngcntr}
\usepackage{graphicx}
\counterwithin*{equation}{section}

%% Title Information.

\title{Assignment 1 \\ OCS}
\author{Marko Ivančić \and Mauro Jurada}
\date{2 November 2019}           %% By default, LaTeX uses the current date

%%%%% The Document

\begin{document}

\maketitle


\section{Characterization of Functions}
%% 1
\begin{equation}
	f(x,y) = 2x^3 - 6y^2 + 3x^2 y
\end{equation}

\includegraphics[width=0.5\textwidth]{Figure_1}
\includegraphics[width=0.5\textwidth]{Surface_1}

$$
\frac {\partial f}{\partial x} = 6x^2 + 6xy \quad\Rightarrow 
	\left\{
	\begin{aligned}
		\frac {\partial^2 f}{\partial x^2}&=12x+6y\\ 
		\frac {\partial^2 f}{\partial y \partial x}&=6x
	 \end{aligned} 
	 \right.
$$

$$
\frac {\partial f}{\partial y} = -12y+3x^2 \quad\Rightarrow 
	\left\{
	\begin{aligned}
		\frac {\partial^2 f}{\partial y^2}&=-12\\ 
		\frac {\partial^2 f}{\partial x \partial y}&=6x
	 \end{aligned} 
	 \right.
$$

Stationary points:

$$
6x^2+6xy=0 \quad\Rightarrow x^2+xy=0 \quad\Rightarrow x(x+y)=0 %% clarification
$$
$$
3x^2-12y=0 \quad\Rightarrow x^2-4y=0
$$
$$
\begin{aligned}
-xy-4y&=0\\
y(x+4)&=0\\
x_1&=0, \quad y_2=0\\
x_2&=-4 \quad\Rightarrow 16-4y=0\\
y_2&=4
\end{aligned}
$$
\begin{gather}
\nabla^2 f(x,y) = 
  \begin{bmatrix}
  12 X + 6Y &
   6X\\
   6X &
   -12 
   \end{bmatrix}
    \nonumber %%no numerating the eq
\end{gather}


$$
\begin{aligned}
\det\left(\nabla^2 f(x,y)\right)=&-12*(12x+6y) - 36x^2= -144x-72y-36x^2\\
1)\: &x_1=0, y_1=0\\
&det=0 \\  \quad\implies \text{Saddle point}%% ??????????????????????????????????????????????????????????????????????????????
2)\: &x_2=-4, y_2=4\\
&det=576-288-576=-288 \quad\implies \text{Saddle point}
\end{aligned}
$$\\

%% 2
\begin{equation}
	f(x,y) = (x - 2y)^4 + 64xy
\end{equation}

\includegraphics[width=0.5\textwidth]{Figure_2}
\includegraphics[width=0.5\textwidth]{Surface_2}

$$
\frac {\partial f}{\partial x} = 4(x-2y)^3 + 64y \quad\Rightarrow 
	\left\{
	\begin{aligned}
		\frac {\partial^2 f}{\partial x^2}&=12(x-2y)\\
		\frac {\partial^2 f}{\partial y \partial x}&=-24(x-2y)^2 +64
	 \end{aligned} 
	 \right.
$$

$$
\frac {\partial f}{\partial y} = -8(x-2y)^3 +64x \quad\Rightarrow 
	\left\{
	\begin{aligned}
		\frac {\partial^2 f}{\partial y^2}&=-48(x-2y)^2\\ 
		\frac {\partial^2 f}{\partial x \partial y}&=-24(x-2)^2 +64
	 \end{aligned} 
	 \right.
$$\\

Stationary points:

$$
4(x-2y)^3 +64y=0  \quad\Rightarrow 
(x-2y)^3 + 16y=0
$$
$$
-8(x-2y)^3+64x=0  \quad\Rightarrow
(x-2y)^3 -8x=0
$$

$$
\begin{aligned}
8x+16y&=0\\
x+2y&=0\\
x&=-2y
\end{aligned}
$$
$$
\begin{aligned}
(-2y -2y)^3 +16y&=0\\
-64y^3 +16y&=0\\
4y^3 -y&=0\\
y(4y^2 -1)&=0\\
y(2y-1)(2y+1)&=0\\
1)\: x_1=&\:0,\quad y_1=0\\
2)\: x_2=&\: -\!1,\quad y_2=\frac{1}{2}\\ 
3)\: x_3=&\:1,\quad y_3=-\frac{1}{2}
\end{aligned}
$$

\begin{gather}
\nabla^2 f(x,y) = 
  \begin{bmatrix}
  12 (X - 2Y)^2 &
  -24 (X - 2Y)^2 + 64\\
  -24 (X - 2Y)^2  +64 &
  48 (X - 2Y)^2 
   \end{bmatrix}
    \nonumber
\end{gather}


$$
\begin{aligned}
\det&\left(\nabla^2 f(x,y)\right)=-12*48(x-2y)^4 -(64-24(x-2y)^2)^2\\
1)\: x_1&=0, \quad y_1=0\\
\det&=12*48(0-0) - (64-24(0-0)^2)^2 =64^2 < 0   \quad\implies \text{Saddle point}\\
2)\: x_2&=-1, \quad y_2=\frac{1}{2}\\
\det&=12*48(-1-1)^4 -(46-24(-1-1)^2)^2=(576*(-16)-32^2)<0 \quad\implies \text{Saddle point}\\
3)\: x_3&=1, \quad y_3=-\frac{1}{2}\\
\det&=12*48*2^4 -(64 -24*2^2)^2 = 576*16-32^2>0 \quad \&\& \quad \frac {\partial^2 f}{\partial x^2} > 0 \quad\implies \text{Global Minimum}\\
\end{aligned}
$$
 \pagebreak

%% 3
\begin{equation}
	f(x,y) = 2x^2 +3y^2 - 2xy + 2x - 3y
\end{equation}

\includegraphics[width=0.5\textwidth]{Figure_3}
\includegraphics[width=0.5\textwidth]{Surface_3}

$$
\frac {\partial f}{\partial x} = 4x-2y+2 \quad\Rightarrow 
	\left\{
	\begin{aligned}
		\frac {\partial^2 f}{\partial x^2}&=4\\
		\frac {\partial^2 f}{\partial y \partial x}&=-2
	 \end{aligned} 
	 \right.
$$
$$
\frac {\partial f}{\partial x} = 6y-2x-3 \quad\Rightarrow 
	\left\{
	\begin{aligned}
		\frac {\partial^2 f}{\partial x^2}&=6\\
		\frac {\partial^2 f}{\partial y \partial x}&=-2
	 \end{aligned} 
	 \right.
$$

Stationary points:

$$
\begin{aligned}
4x-2y+2&=0 \quad\Rightarrow 2x-y+1=0 \quad\Rightarrow y=2x+1\\
6y-2x-3&=0\\
&6(2x+1) -2x-3=0\\
&10x-3=0\\
&x=\frac{-3}{10}\\
&y=2\frac{-3}{10} +1 = \frac{2}{5}
\end{aligned}
$$

\begin{gather}
\nabla^2 f(x,y) = 
  \begin{bmatrix}
   4 &
   -2\\
   -2 &
   6 
   \end{bmatrix}
   \nonumber
\end{gather}

$$
\det\left(\nabla^2 f(x,y)\right) = 4*6-(-2)^2=24-4=20 >0  \quad \&\& \quad 4 > 0 \quad\implies \text{Strict Global Minimum}\\ % ???????????????
$$\\


%% 4
\begin{equation}
	f(x,y) = \ln(1 + \dfrac{1}{2}(x^2 +3y^2))
\end{equation}

\includegraphics[width=0.5\textwidth]{Figure_4}
\includegraphics[width=0.5\textwidth]{Surface_4}

$$
\begin{aligned}
\frac {\partial f}{\partial x} = \frac{1}{1+\frac{1}{2}(x^2+3y^2)}x= \frac{x}{\frac{1}{2}(2+x^2+3y^2)}=\frac{2x}{2+x^2+3y^2}\\
\implies 
	\left\{
	\begin{aligned}
		\frac {\partial^2 f}{\partial x^2}&=\frac{2(2+x^2+3y^2)-2x(2x)}{(2+x^2+3y^2)^2}=
		 	\frac{4-2x^2+6y^2}{(2+x^2+3y^2)^2} \\
		\frac {\partial^2 f}{\partial y \partial x}&=\frac{0(2+x^2+3y^2)-2x(6y)}{(2+x^2+3y^2)^2}=
		 	\frac{-12xy}{(2+x^2+3y^2)^2} 
	 \end{aligned} 
	 \right.
\end{aligned}
$$

$$
\begin{aligned}
\frac {\partial f}{\partial x} = \frac{1}{1+\frac{1}{2}(x^2+3y^2)}3y= \frac{3y}{\frac{1}{2}(2+x^2+3y^2)}=\frac{6y}{2+x^2+3y^2}\\
\implies 
	\left\{
	\begin{aligned}
		\frac {\partial^2 f}{\partial x^2}&=\frac{6(2+x^2+3y^2)-6y(6y)}{(2+x^2+3y^2)^2}=
		 	\frac{12+6x^2-18y^2}{(2+x^2+3y^2)^2} \\
		\frac {\partial^2 f}{\partial y \partial x}&=\frac{0(2+x^2+3y^2)-6y(2x)}{(2+x^2+3y^2)^2}=
		 	\frac{-12xy}{(2+x^2+3y^2)^2} 
	 \end{aligned} 
	 \right.
\end{aligned}
$$

Stationary points:

$$
\frac{2x}{2+x^2+3y^2}=0 \quad\Rightarrow x=0\\
$$
$$
\frac{6y}{2+x^2+3y^2}=0 \quad\Rightarrow y=0\\
$$
\begin{gather}
\nabla^2 f(x,y) = 
  \begin{bmatrix}
   \dfrac{4 - 2X^2 +6Y^2}{(2 + X^2 + 3Y^2)^2} &
   \dfrac{-12XY}{(2 + X^2 + 3Y^2)^2}\\
   \dfrac{-12XY}{(2 + X^2 + 3Y^2)^2} &
   \dfrac{12 + 6X^2 - 18Y^2}{(2 + X^2 + 3Y^2)^2} 
   \end{bmatrix}
    \nonumber
\end{gather}

$$
\det\left(\nabla^2 f(x,y)\right)=\frac{4-2x^2+6y^2}{(2+x^2+3y^2)^2} * \frac{12+6x^2-18y^2}{(2+x^2+3y^2)^2} - \frac{-12xy}{(2+x^2+3y^2)^2} *  \frac{-12xy}{(2+x^2+3y^2)^2}
$$
$$
1) x=0, \quad y=0 \quad\Rightarrow det=\frac{4}{4}*\frac{12}{4}-\frac{0}{4}*\frac{0}{4} > 0  \quad \&\& \quad \frac{4}{4} > 0 \quad\implies \text{Strict Global Minimum}
$$

%Consider the two points $(-1,16)$ and $(3,1)$.  quation $y = m x + b$ through the two points, and
%Section fits a exponential equation $y = A e^{k x}$
% $y$-intercept $b$ of the line.


\section{Numerical Gradient Approximation}


\begin{equation}
	\nabla_x f(x,y) \approx \dfrac{f(x + \epsilon, y) - f(x - \epsilon, y)}{2\epsilon}
\end{equation}

\begin{equation}
	\nabla_x f(x,y) \approx \dfrac{f(x, y + \epsilon) - f(x, y - \epsilon)}{2\epsilon}
\end{equation}

1.1)\\
$$
\begin{aligned}
(x,y) &= (1, -4)\\
\nabla_x f(x,y) &= -17.999998000007622\\
\nabla_y f(x,y) &= 51.00000000000904\\
\frac {\partial f}{\partial x} &= -18\\
\frac {\partial f}{\partial y} &= 51\\\\
\end{aligned}
$$

1.2)\\
$$
\begin{aligned}
(x,y) &= (0, 2)\\
\nabla_x f(x,y) &= -128.00001600004407\\
\nabla_y f(x,y) &= 512.00012799994\\
\frac {\partial f}{\partial x} &= -128\\
\frac {\partial f}{\partial y} &= 512\\\\
\end{aligned}
$$

1.3) \\
$$
\begin{aligned}
(x,y) &= (1, 3)\\
\nabla_x f(x,y) &= -1.7763568394002505e-12\\
\nabla_y f(x,y) &= 12.9999999999999\\
\frac {\partial f}{\partial x} &= 0\\
\frac {\partial f}{\partial y} &= 13\\\\
\end{aligned}
$$

1.4)\\
$$
\begin{aligned}
(x,y) &= (-4, 4)\\
\nabla_x f(x,y) &= -0.12121211996918291\\
\nabla_y f(x,y) &= 0.3636363631356332\\
\frac {\partial f}{\partial x} &= -0.12121212\\
\frac {\partial f}{\partial y} &= 0.363636363\\\\
\end{aligned}
$$


\section{Vectors, Norms and Matrices}


\subsection{}
$\vert$$\vert$.$\vert$$\vert_{1/2}$ is not a norm because it violates the third rule for norms, triangle inequality, for vectors x=(1,0) and y=(0,1)

$$
\begin{aligned}
\Vert x+y\Vert &\leq\Vert x\Vert  +\Vert y\Vert \\
\Vert (1, 0) + (0, 1)\Vert &\leq \Vert (1, 0)\Vert  + \Vert (0, 1)\Vert \\
\Vert (1, 1)\Vert &\leq \Vert (1, 0)\Vert  + \Vert (0, 1)\Vert \\
(\sqrt{1} + \sqrt{1})^2&\leq (\sqrt{1} + \sqrt{0})^2 + (\sqrt{0} + \sqrt{1})^2\\
2^2 &\leq 1^2 + 1^2\\
4 &\leq 2\\
\end{aligned}
$$

\subsection{}
To prove this inequality we will use the triangle inequality
$$
\begin{aligned}
||a + b||&\leq ||a|| + ||b||\\
\text{Replace a and b}\\ a &= x-y\\ b &= y-z\\ \text{then}\\
||x - y + y - z||&\leq ||x-y|| + ||y-z||\\
||x - z||&\leq ||x-y|| + ||y-z||\\
\end{aligned}
$$





\section{Matrix Calculus}

\begin{equation}
	f(x) = \dfrac{1}{2}\Vert Ax - b\Vert^2 \quad\text{for}\; x\in \mathbb{R} ^n ,b\in \mathbb{R}^m \;\text{and}\; A\in\mathbb{R}^{m*n}
\end{equation}


$$
\begin{aligned}
\text{We write the equation in vector form:}\\
f(x) = \dfrac{1}{2}\left\Vert
  \begin{bmatrix}
 \sum_{j=0}^{n}a_{0j}x_j - b_0\\
\vdots\\
  \sum_{j=0}^{n}a_{mj}x_j - b_m
   \end{bmatrix}
\right\Vert^2\\\\
\text{The square root is cancelled by power of 2}\\
f(x) = \dfrac{1}{2}\sum_{i=0}^{m}\left(\sum_{j=0}^{n}a_{ij}x_j - b_i\right)^2\\\\
\text{we derive the bracket and whats inside the bracket}\\
\frac {\partial f(x)}{\partial x_k} = \sum_{i=0}^{m}\left(\sum_{j=0}^{n}a_{ij}x_j - b_i\right)a_{ik}\\
\end{aligned}
$$
$$
\text{Linear form: }\\
	\frac {\partial f(x)}{\partial x}  = A^\top(Ax - b)
$$\\
\begin{equation}
	f(\alpha) = \dfrac{1}{2}||A(x - \alpha y) - b||^2
\end{equation}
\\
\\
$$
\begin{aligned}
\text{We write the equation in vector form:}\\
f(x) = \dfrac{1}{2}\left\Vert
  \begin{bmatrix}
 \sum_{j=0}^{n}a_{0j}(x_j + \alpha y_j) - b_0\\
\vdots\\
  \sum_{j=0}^{n}a_{mj}(x_j + \alpha y_j) - b_m
   \end{bmatrix}
\right\Vert^2\\\\
\text{The square root is cancelled by power of 2}\\
f(x) = \dfrac{1}{2}\sum_{i=0}^{m}\left(\sum_{j=0}^{n}a_{ij}(x_j + \alpha y_j) - b_i\right)^2\\\\
\text{we derive the bracket and whats inside the bracket}\\
\frac {\partial f(\alpha)}{\partial \alpha} = \sum_{i=0}^{m}\left[ \left( \sum_{j=0}^{n}(a_{ij}x_j +a_{ij}y_j\alpha) - b_i\right)\sum_{j=0}^{n}a_{ij}y_j\right]\\
\end{aligned}
$$
$$
\text{Linear form: }\\
	\frac {\partial f(\alpha)}{\partial \alpha} = y^\top A^\top(A(x +\alpha y) - b)
$$\\

%%$$
	%%16 = A e^{(-\ln(2))(-1)} = A e^{\ln{2}} = 2 A
	.
%%$$


\section{Student Task Selection Problem}

Two students need to complete an assignment sheet, which consists of $J= 15$ tasks.  Each student has a total time budget which can be assigned to the individual tasks. To account for the different abilities of the students, they estimated for each student the required time for each task in hours as well as their time budget and inserted it into the following table:

\begin{table}[h]
\resizebox{\textwidth}{!}{%
\begin{tabular}{l|l|l|l|l|l|l|l|l|l|l|l|l|l|l|l|l}
        & \multicolumn{15}{l|}{Task}                                                                       &             \\ \cline{1-16}
Student & 1    & 2    & 3    & 4    & 5   & 6    & 7     & 8    & 9   & 10   & 11  & 12  & 13  & 14  & 15  & Time budget \\ \hline
1       & 0.5  & 0.25 & 0.25 & 0.25 & 1.0 & 1.0  & 0.5   & 0.5  & 1.0 & 0.5  & 1.5 & 2.5 & 1.0 & 2.5 & 3.5 & 9           \\ \cline{1-16}
2       & 0.75 & 1.0  & 0.75 & 0.5  & 0.5 & 0.25 & 0.25 & 0.25 & 2.0 & 1.25 & 1.0 & 4.0 & 2.5 & 3.0 & 2.0 & 6          
\end{tabular}%
}
\end{table}

To figure out the distribution of students to tasks that minimizes the time to complete the whole assignment sheet, first the student task selection problem needs to be cast into a linear program. $x$ values will define whether the first or the second student  $(x_1, x_{15})$ and $(x_{16}, x_{30})$, respectively, does the individual task.
 
 $$
 \begin{aligned}
 \min_{(x_1,x_{30})\in[0 .. 1]} &= 0.5x_1 + 0.25x_2 + 0.25x_3 + 0.25x_4 + 1.0x_5 + 1.0x_6+ 0.5x_7 + 0.5x_8\\&+ 1.0x_9 +0.5x_{10}+ 1.5x_{11}+ 2.5x_{12}+ 1.0x_{13}+ 2.5x_{14}+ 3.5x_{15}+
 0.75x_{16}\\ &+ 1.0x_{17}+  0.75x_{18}+ 0.5 x_{19}+ 0.5x_{20}+ 0.25x_{21}+ 0.25x_{22}+ 0.25x_{23}\\ &+2.0x_{24}+ 1.25x_{25}+1.0x_{26}+ 4.0x_{27}+ 2.5x_{28}+ 3.0 x_{29}+ 2.0x_{30}
 \end{aligned}
$$

s.t.

\begin{equation}
 \sum_{i=0}^{15} ( c_i x_i) \le 9 
\end{equation}
\begin{equation}
\sum_{i=16}^{30} ( c_i x_i) \le 6 
\end{equation}

\noindent where $c_i$ are the coefficients. To convert into the standard form we introduce $s_1$ and $s_2$ which measure how much under $9$ and $6$ the constraint functions are.

$$
 \sum_{i=0}^{15} ( c_i x_i) + s_1 = 9 \quad\Rightarrow
 \sum_{i=0}^{15} ( c_i x_i) + s_1 - 9 = 0
$$
$$
 \sum_{i=0}^{15} ( c_i x_i) + s_2 = 6\quad\Rightarrow
 \sum_{i=0}^{15} ( c_i x_i) + s_2 - 6 = 0
$$
$$
s_1, s_2 \ge 0, \quad x_i \in [0 .. 1]
$$

Python scipy.optimize.linprog %% insert reference
will be used to minimize a linear objective function subject to linear equality and inequality constraints. \\
Inequality constraints are the eq. (1,2). Equality constraint is $x_i + x_{i+15} = 1$ because any one task can only be done by one student.
\\ \\
Solution:

\begin{table}[h]
\resizebox{\textwidth}{!}{%
\begin{tabular}{l|lllllllllllllllll}
        & \multicolumn{15}{l|}{Task}                                                                                                                                                                                                                                                                                                                                                                 &                                  &           \\ \cline{1-16}
Student & \multicolumn{1}{l|}{1} & \multicolumn{1}{l|}{2} & \multicolumn{1}{l|}{3} & \multicolumn{1}{l|}{4} & \multicolumn{1}{l|}{5} & \multicolumn{1}{l|}{6} & \multicolumn{1}{l|}{7} & \multicolumn{1}{l|}{8} & \multicolumn{1}{l|}{9} & \multicolumn{1}{l|}{10} & \multicolumn{1}{l|}{11} & \multicolumn{1}{l|}{12} & \multicolumn{1}{l|}{13} & \multicolumn{1}{l|}{14} & \multicolumn{1}{l|}{15} & \multicolumn{1}{c|}{Time budget} & Minimized \\ \hline
1       & \multicolumn{1}{l|}{1} & \multicolumn{1}{l|}{1} & \multicolumn{1}{l|}{1} & \multicolumn{1}{l|}{1} & \multicolumn{1}{l|}{0} & \multicolumn{1}{l|}{0} & \multicolumn{1}{l|}{0} & \multicolumn{1}{l|}{0} & \multicolumn{1}{l|}{1} & \multicolumn{1}{l|}{1}  & \multicolumn{1}{l|}{0}  & \multicolumn{1}{l|}{1}  & \multicolumn{1}{l|}{1}  & \multicolumn{1}{l|}{1}  & \multicolumn{1}{l|}{0}  & \multicolumn{1}{l|}{9}           & 8.75      \\ \hline
2       & \multicolumn{1}{l|}{0} & \multicolumn{1}{l|}{0} & \multicolumn{1}{l|}{0} & \multicolumn{1}{l|}{0} & \multicolumn{1}{l|}{1} & \multicolumn{1}{l|}{1} & \multicolumn{1}{l|}{1} & \multicolumn{1}{l|}{1} & \multicolumn{1}{l|}{0} & \multicolumn{1}{l|}{0}  & \multicolumn{1}{l|}{1}  & \multicolumn{1}{l|}{0}  & \multicolumn{1}{l|}{0}  & \multicolumn{1}{l|}{0}  & \multicolumn{1}{l|}{1}  & \multicolumn{1}{l|}{6}           & 4.25      \\ \hline
Total   &                        &                        &                        &                        &                        &                        &                        &                        &                        &                         &                         &                         &                         &                         &                         & \multicolumn{1}{l|}{}            & 13       
\end{tabular}%
}
\end{table}


\end{document}

